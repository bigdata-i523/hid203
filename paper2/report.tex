\documentclass[sigconf]{acmart}

\input{format/i523}

\begin{document}
\title{Big Data for Edge Computing}


\author{Ben Trovato}
\orcid{1234-5678-9012}
\affiliation{%
  \institution{Institute for Clarity in Documentation}
  \streetaddress{P.O. Box 1212}
  \city{Dublin} 
  \state{Ohio} 
  \postcode{43017-6221}
}
\email{trovato@corporation.com}

\author{G.K.M. Tobin}
\affiliation{%
  \institution{Institute for Clarity in Documentation}
  \streetaddress{P.O. Box 1212}
  \city{Dublin} 
  \state{Ohio} 
  \postcode{43017-6221}
}
\email{webmaster@marysville-ohio.com}

\author{Gregor von Laszewski}
\affiliation{%
  \institution{Indiana University}
  \streetaddress{Smith Research Center}
  \city{Bloomington} 
  \state{IN} 
  \postcode{47408}
  \country{USA}}
\email{laszewski@gmail.com}


% The default list of authors is too long for headers}
\renewcommand{\shortauthors}{G. v. Laszewski}


\begin{abstract}
This paper provides a sample of a \LaTeX\ document which conforms,
somewhat loosely, to the formatting guidelines for
ACM SIG Proceedings.
\end{abstract}

\keywords{Big Data, Edge Computing i523}


\maketitle



\section{Introduction}

Put here an introduction about your topic. 
We just need one sample refernce so the paper compiles in LaTeX so we
put it here \cite{editor00}.

\section{Big Data and Statistics}

Put here an introduction about your topic. 
We just need one sample refernce so the paper compiles in LaTeX so we
put it here \cite{editor00}.


\section{Traditional Predictive Modeling}

Predictive modeling is one of the most important type of statistical analysis of big data. It aims to model the causal relationship between different features present in the data. Specifically, we try to predict the value of one variable, known as the dependent or response variable, with the help of one or more variables, known as independent or explanatory variables. The traditional predictive modeling process is shown in Figure 1.

\begin{figure}[!ht]
  \centering\includegraphics[width=\columnwidth]{images/fly.pdf}
  \caption{Example caption}\label{f:fly}
\end{figure}

The steps in traditional predictive modeling are: \cite{part-reg}

\subsection{Define Goal}
For any predictive modeling, it is important to clearly define the goal, i.e., define the response variable and the explanatory variables that we are going to use to predict the response variable.
\subsection{Data Collection and Management} 
This is another critical step for predictive modeling which requires identifying the data that can be used for analysis. It can be the most time consuming step in the entire modeling process and may require some preliminary data exploration and visualization. It also involves identifying the feature set and the structure of each feature.
\subsection{Data Preprocessing} 
Before building any predictive model, it is important to perform data quality checks. Two major components of data preprocessing are 
\begin{itemize}
	\item Analyzing missing values: For missing values, it is important to identify whether we want to drop the missing entries in the data or impute them using standard imputation methods such as mean imputation for quantitative features and mode imputation for qualitative features.
	\item Data transformation: The aim of data transformation is to convert it into a form which is easier to model. For example, normalization and standardization of data helps in better interpretation of the co-efficients of regression model. Some of the transformations also depend on the predictive model that we intend to apply. For example, for a linear regression model, it is important to ensure that the dependent and the independent variables have a linear relationship and that the dependent variable has a constant variance.
\end{itemize}
\subsection{Exploratory Data Analysis (EDA)} 
As part of EDA, we try to summarize the data graphically and analyze each feature along with the relationship between different features. For summarization, a variety of summary statistics such as mean, median, variance, etc. are used. In case of predictive modeling, one might want to explore the data and visualize the numerical summary along with the correlation of different features in the data set.
\subsection{Model-building and reporting}
The final step involves building the predictive model using the clean transformed data. Different regression techniques such as linear regression can be employed for predicting the response variable from the independent variables. Once the model is built, it is evaluated using different metrics like co-efficient of determination.

\section{Predictive Modeling with Big Data}

Put here an introduction about your topic. 
We just need one sample refernce so the paper compiles in LaTeX so we
put it here \cite{editor00}.

\subsection{Illustration}

Put here an introduction about your topic. 
We just need one sample refernce so the paper compiles in LaTeX so we
put it here \cite{editor00}.

\section{Big Data Systems and their support in R}

Put here an introduction about your topic. 
We just need one sample refernce so the paper compiles in LaTeX so we
put it here \cite{editor00}.

\section{Regression Techniques for Big Data}

Put here an introduction about your topic. 
We just need one sample refernce so the paper compiles in LaTeX so we
put it here \cite{editor00}.

\section{figures}

In Figure \ref{f:fly} we show a fly. Please note that because we use
just columwidth that the size of the figure will change to the
columnwidth of the paper once we change the layout to final. CHnaging
the layout to final should not be done by you. All figures will be
listed at the end.

\begin{figure}[!ht]
  \centering\includegraphics[width=\columnwidth]{images/fly.pdf}
  \caption{Example caption}\label{f:fly}
\end{figure}

When copying the example, please do not check in the images from the
examples into your images directory as you will not need them for your
paper. Instead use images that you like to include. If you do not have
any images, do not dreate the images folder.

\section{Tables}

In case you need to create tables, you can do this with online tools
(if you do not mind sharing your data) such as
\url{https://www.tablesgenerator.com/} or other such tools (please
google for them). They even allow you to manage tables as CSV.

or generate them by hand while using the provided template in Table\ref{t:mytable}. Not ethat
the caption is before the tabular environment.

\begin{table}[htb]
\centering
\caption{My caption}
\label{t:mytabble}
\begin{tabular}{lll}
1 & 2 & 3 \\
\hline
4 & 5 & 6 \\
7 & 8 & 9
\end{tabular}
\end{table}

\section{Long example}

If you like to see a more elaborate example, please look at
report-long.tex. 

\section{Conclusion}

Put here an conclusion. Conlcusions and abstracts must not have any
citations in the section.


\begin{acks}

  The authors would like to thank Dr. Gregor von Laszewski for his
  support and suggestions to write this paper.

\end{acks}

\bibliographystyle{ACM-Reference-Format}
\bibliography{report} 


\end{document}
